%Copyright 2021 Jean-Michel Bruel
%This program is free template, under the terms of the attached License of this repo. 
%Based on https://github.com/jpeisenbarth/SRS-Tex
%Itself based on the code of Yiannis Lazarides
%http://tex.stackexchange.com/questions/42602/software-requirements-specification-with-latex
%http://tex.stackexchange.com/users/963/yiannis-lazarides
%Also based on the template of Karl E. Wiegers
%http://www.se.rit.edu/~emad/teaching/slides/srs_template_sep14.pdf
%http://karlwiegers.com
\documentclass{scrreprt}
\usepackage{method}		% Specific command from Handbook
\usepackage{minitoc}	% For table of content in chapters
\usepackage{listings}
\usepackage{underscore}
\usepackage{xcolor}
\usepackage{booktabs}
\usepackage[bookmarks=true]{hyperref}
\usepackage[utf8]{inputenc}
\usepackage[english]{babel}
\hypersetup{
    bookmarks=false,    % show bookmarks bar?
    pdftitle={Software Requirement Specification},    % title
    pdfauthor={Jean-Philippe Eisenbarth},                     % author
    pdfsubject={TeX and LaTeX},                        % subject of the document
    pdfkeywords={TeX, LaTeX, graphics, images}, % list of keywords
    colorlinks=true,       % false: boxed links; true: colored links
    linkcolor=blue,       % color of internal links
    citecolor=black,       % color of links to bibliography
    filecolor=black,        % color of file links
    urlcolor=purple,        % color of external links
    linktoc=page            % only page is linked
}%
\def\myversion{1.0.0 }
\date{}
%\title
\usepackage{hyperref}
\newcommand*{\nsection}[1]{
    \section*{#1}
    \addcontentsline{toc}{section}{#1}
}
\newcommand*{\nsubsection}[1]{
    \subsection*{#1}
    \addcontentsline{toc}{section}{#1}
}
%--------------------- Numbering ---------------
\usepackage{amsthm}
\usepackage{xassoccnt}
%\newtheorem{req}{Requirement}[section]        
\newtheorem{req}{Requirement}        
\theoremstyle{definition}        
\newtheorem{constraint}{Constraint}
\newtheorem{goal}{Goal}
\newtheorem{limitation}{Limitation}
\newtheorem{exclusion}{Exclusion}
\DeclareCoupledCountersGroup{theorems}
\DeclareCoupledCounters[name=theorems]{req,constraint,goal}
\setcounter{goal}{0}
%--------------------- Comments ---------------
\newcommand{\mynote}[3][black]{\textcolor{#1}{\fbox{\bfseries\sffamily\scriptsize{#2}}
{\small\textsf{\emph{#3}}}}}
\newcommand{\comment}[1]{\mynote[red]{Comment:}{#1}}
%\newcommand{\comment}[1]{}
\newcommand{\notEmpty}{\comment{This chapter should not be empty!}}

\begin{document}

\begin{center}
    \rule{16cm}{5pt}\vskip1cm
    \begin{bfseries}
        \Huge{SOFTWARE REQUIREMENTS\\ SPECIFICATION}\\
        \vspace{1.9cm}
        Room8\\
        \vspace{1.9cm}
        \LARGE{Version \myversion}\\
        \vspace{1.9cm}
        Prepared by:\\
        Mohammed Abed\\ 
        Maged Armanios\\
        Jinal Kasturiarachchi\\
        Jane Klavir\\
        Harshil Patel\\
        \vspace{0.9cm}
        McMaster University\\
        \vspace{0.9cm}
        \today\\
    \end{bfseries}
\end{center}

%===============================================================
\chapter*{Revision History}
%===============================================================

\begin{center}
    \begin{tabular}{|c|c|c|c|}
        \hline
	    Name        & Date          & Reason For Changes & Version\\
        \hline
        \hline
	    J.-M. Bruel & 2021-01-22    & First Draft & 1.0\\
        \hline
	    J.-M. Bruel & 2023-01-28    & Check after publication of the Handbook & 1.23\\
        \hline
	    J.-M. Bruel & 2023-06-12    & Add reqs automated numbering & 1.23.1 \\
        \hline
	    J.-M. Bruel & 2023-08-25    & Add Minimum Requirements Outcome Principle & 1.23.8 \\
        \hline
	    J.-M. Bruel & 2023-12-22    & Remove section numbers & 1.23.12 \\
        \hline
	    J.-M. Bruel & 2024-08-01    & Add warning about non empty chapters  & \myversion \\
        \hline
    \end{tabular}
\end{center}

This document follows the requirements documentation structure presented in the \href{ https://link.springer.com/content/pdf/10.1007/978-3-031-06739-6.pdf}{Handbook of requirements and business analysis}, by Bertrand Meyer.

%===============================================================
\dominitoc% Initialization
\tableofcontents
\adjustmtc
%===============================================================

%===============================================================
\addcontentsline{toc}{chapter}{Goals book}
\chapter*{Goals}

\minitoc% Creating an actual minitoc
%===============================================================

\comment{Goals are "needs of the target organization, which the system will address". While the development team is the principal user of the other books, the Goals book addresses a wider audience: essentially, all stakeholders.}

\nsection{G.1 Context and overall objective}
\comment{High-level view of the project: organizational context and reason for building a system.} 
%---------------------------------
\notEmpty{}
%---------------------------------

\begin{goal}\label{goal:first}
This is a goal example. If you need explicit (and automatic) numbering, you can use the definitions in the \texttt{.tex} template.
Is is refined by \ref{req:memo}
\end{goal}

\begin{req}\label{req:cross}
This is a requirement example. It illustrates how numbering is continuous and cross-types (if this is what you need).
\end{req}
    
\nsection{G.2 Current situation}
\comment{Current state of processes to be addressed by the project and the resulting system.}

\begin{req}\label{req:memo}
This is a requirement example. 
It refines \ref{goal:first}
\end{req}
    
\nsection{G.3 Expected benefits}
\comment{New processes, or improvement to existing processes, made possible by the project’s results.}
%---------------------------------
\notEmpty{}
%---------------------------------

\nsection{G.4 Functionality overview}
\comment{Overview of the functions (behavior) of the system. Principal properties only (details are in the System book).}

\nsection{G.5 High-level usage scenarios}
\comment{Fundamental usage paths through the system.}


\nsection{G.6 Limitations and exclusions}
\comment{Aspects that the system need not address.}
Below is a list of limitations and exclusions the system will not address:
\begin{limitation}\label{limitation:first}
System will not track activity completed by the user in the shared environment.
\end{limitation}
\begin{exclusion}\label{exclusion:first}
System will not request or send money directly to users in the bill splitting functionality.
\end{exclusion}
\begin{exclusion}\label{exclusion:second}
System will not use images taken to train machine learning model.
\end{exclusion}

\nsection{G.7 Stakeholders and requirements sources}
\comment{Groups of people who can affect the project or be affected by it, and other places to consider for information about the project and system.}
%---------------------------------
\notEmpty{}
%---------------------------------

\begin{table}[ht]
\centering
\begin{tabular}{|l|l|}
\hline
\textbf{Stakeholder} & \textbf{Category} \\ \hline
Students              & Direct  \\ \hline
Home Managers         & Indirect \\ \hline
University Housing and Social Committee & Indirect \\ \hline
\end{tabular}
\caption{Stakeholders and Categories}
\end{table}

\nsubsection{G.7.1 Direct Stakeholders}
\textbf{Students} \\ Students are the primary direct stakeholders for Room8. They are the main users of the mobile application who create houses within the app and set up the camera systems. These students seek to maintain cleanliness in their shared living spaces and establish accountability when a mess is left behind by a roommate. The application addresses common challenges faced by students in shared living arrangements, such as maintaining cleanliness, splitting expenses, and scheduling activities. 

\nsubsection{G.7.1 Indirect Stakeholders}
\textbf{Home Managers} \\ Home managers are indirect stakeholders to the project. Home managers includes landlords renting out their homes to students or residence assistants managing a room of students. The home managers look to maintain the clean condition of the shared space which is done by holding students accountable for messes that are made. 
\\
\\
\textbf{University Housing and Social Committee} \\ The University Housing and Social Committee is another key indirect stakeholder in the project. These committees often seek to help students transition into living in shared spaces and provide guidance and support. Room8 offers a wide range of services that address common points of frustration faced by students, which are often brought up to these university committees. By facilitating better communication and organization, Room8 helps enhance the overall living experience for students.



%===============================================================
\chapter*{Environment}
\addcontentsline{toc}{chapter}{Environment book}
\minitoc% Creating an actual minitoc
%===============================================================

\comment{The Environment book describes the application domain and external context, physical or virtual (or a mix), in which the system will operate.}

\nsection{E.1 Glossary}
\comment{Clear and precise definitions of all the vocabulary specific to the application domain, including technical terms, words from ordinary language used in a special meaning, and acronyms.
This chapter should not be empty!}

\nsection{E.2 Components}
\comment{List of elements of the environment that may affect or be affected by the system and project. Includes other systems to which the system must be interfaced.}

\nsection{E.3 Constraints}
\comment{Obligations and limits imposed on the project and system by the environment.}
%---------------------------------
\notEmpty{}
%---------------------------------


\nsection{E.4 Assumptions}
\comment{Properties of the environment that may be assumed, with the goal of facilitating the project and simplifying the system.}

\nsection{E.5 Effects}
\comment{Elements and properties of the environment that the system will affect.}

\nsection{E.6 Invariants}
\comment{Properties of the environment that the system’s operation must preserve.}

%===============================================================
\chapter*{System}
\addcontentsline{toc}{chapter}{System book}
\minitoc% Creating an actual minitoc
%===============================================================

\comment{The System book refines the Goal one by focusing on more detailed requirements about the system under development, mainly its constituents, behaviors and properties.}

\nsection{S.1 Components}
\comment{Overall structure expressed by the list of major software and, if applicable, hardware parts.}
%---------------------------------
\notEmpty{}
%---------------------------------

\nsection{S.2 Functionality}
\comment{One section, S.2.n, for each of the components identified in S.2, describing the corresponding behaviors (functional and non-functional properties).}
%---------------------------------
\notEmpty{}
%---------------------------------

\nsection{S.3 Interfaces}
\comment{How the system makes the functionality of S.2 available to the rest of the world, particularly user interfaces and program interfaces (APIs).}

\nsection{S.4 Detailed usage scenarios}
\comment{Examples of interaction between the environment (or human users) and the system: use cases, user stories.}

\nsection{S.5 Prioritization}
\comment{Classification of the behaviors, interfaces and scenarios (S.2, S.3 and S.4) by their degree of criticality.}

\nsection{S.6 Verification and acceptance criteria}
\comment{Specification of the conditions under which an implementation will be deemed satisfactory.}

%===============================================================
\chapter*{Project}
\addcontentsline{toc}{chapter}{Project book}
\minitoc% Creating an actual minitoc
%===============================================================

\comment{The Project book describes all the constraints and expectations not about the system itself, but about how to develop and produce it.}

\nsection{P.1 Roles and personnel}
\comment{Main responsibilities in the project; required project staff and their needed qualifications.}

\nsection{P.2 Imposed technical choices}
\comment{Any a priori choices binding the project to specific tools, hardware, languages or other technical parameters.}

\nsection{P.3 Schedule and milestones}
\comment{List of tasks to be carried out and their scheduling.}
%---------------------------------
\notEmpty{}
%---------------------------------

\nsection{P.4 Tasks and deliverables}
\comment{Details of individual tasks listed under P.3 and their expected outcomes.}
%---------------------------------
\notEmpty{}
%---------------------------------

\nsection{P.5 Required technology elements}
\comment{External systems, hardware and software, expected to be necessary for building the system.}

\nsection{P.6 Risks and mitigation analysis}
\comment{Potential obstacles to meeting the schedule of P.4, and measures for adapting the plan if they do arise.}

\nsection{P.7 Requirements process and report}
\comment{Initially, description of what the requirements process will be; later, report on its steps.}

\newpage{}
\section*{Appendix --- Reflection}

The information in this section will be used to evaluate the team members on the
graduate attribute of Lifelong Learning.  

The purpose of reflection questions is to give you a chance to assess your own
learning and that of your group as a whole, and to find ways to improve in the
future. Reflection is an important part of the learning process.  Reflection is
also an essential component of a successful software development process.  

Reflections are most interesting and useful when they're honest, even if the
stories they tell are imperfect. You will be marked based on your depth of
thought and analysis, and not based on the content of the reflections
themselves. Thus, for full marks we encourage you to answer openly and honestly
and to avoid simply writing ``what you think the evaluator wants to hear.''

Please answer the following questions.  Some questions can be answered on the
team level, but where appropriate, each team member should write their own
response:


\begin{enumerate}
  \item What went well while writing this deliverable? 
  \item What pain points did you experience during this deliverable, and how did
  you resolve them?
  \item How many of your requirements were inspired by speaking to your
  client(s) or their proxies (e.g. your peers, stakeholders, potential users)?
  \item Which of the courses you have taken, or are currently taking, will help
  your team to be successful with your capstone project.
  \item What knowledge and skills will the team collectively need to acquire to
  successfully complete this capstone project?  Examples of possible knowledge
  to acquire include domain specific knowledge from the domain of your
  application, or software engineering knowledge, mechatronics knowledge or
  computer science knowledge.  Skills may be related to technology, or writing,
  or presentation, or team management, etc.  You should look to identify at
  least one item for each team member.
  \item For each of the knowledge areas and skills identified in the previous
  question, what are at least two approaches to acquiring the knowledge or
  mastering the skill?  Of the identified approaches, which will each team
  member pursue, and why did they make this choice?
\end{enumerate}


%===============================================================
\end{document}
%===============================================================